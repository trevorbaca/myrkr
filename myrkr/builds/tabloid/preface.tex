\documentclass[10pt]{article}
\usepackage[utf8]{inputenc}
\usepackage[papersize={17in, 11in}]{geometry}
\usepackage[absolute]{textpos}
\TPGrid[0.5in, 0.25in]{23}{24}
\usepackage{palatino}
\usepackage{amsmath}
\parindent=0pt
\parskip=12pt
\usepackage{nopageno}
\usepackage{pifont}
\begin{document}

\begin{textblock}{11.5}(5.75, 6)

\textit{Architecture of darkness: footfall and then warmth. Around us he draws
the sheets. To both lover and beloved night extends a hand: she leads us,
docent in the house of dreams, to the terra incognita of our sleep. We
understand better in this place, unknown on our maps, where sight takes its
leave and where objects recede: nightfall and touch and our comprehension (all
at once) of skin and the dark and its shapes.}

``Myrkr'' is the Old Norse for ``darkness.'' The piece proposes pathways into
sleep. Five courses crisscross the music in voices distinguished by colors
special to the instrument and by the jittery durations used in the lines'
animation. The trajectory of the piece sinks slowly downward~---~dark flowers
garlanded together~---~until memory releases the music into morning.

\textbf{Interpretation.} The music is structured according to an interpolation
of voices. The interpretation of this type of intercalative polyphony should
prioritize the connectedness of each voice: one voice interrupting another
should convey a type of curtailment; each voice's reappearance should effect
the sudden return of a music hidden from view. \textbf{Dynamics.} Play all
changes of dynamic subito. Changes of dynamic correspond to voice reentries.
\textbf{Tonguing.} Decisions of tonguing are left to the performer. Because of
this no slurs appear in the score. \textbf{Glissandi.} Do not articulate
pitches internal to glissandi. \textbf{Color fingerings.} Play fingerings
\ding{172}, \ding{173}, \ding{174}, \ding{175} as increasingly different
versions of the pitches over which they appear. Chose fingerings that minimize
differences in pitch while maximizing differences in color. \textbf{Overblown
multiphonics.} Play the notes marked ``overblow'' as aggressive multiphonics;
all such notes accompany loud dynamics. The overtone content of all such notes
is left to the performer. \textbf{Vowel colors.} Indications of vowel color
(``A'', ``E'', ``I'', ``O'', ``U'``) appear in the last two sections of the
piece. Shape the inside of the vocal cavity according to the vowels indicated
to emphasize upper partials of the fundamental in a hautingly quiet way. Do not
substitute multiphonics for changes of vowel color. \textbf{Other notes.}
Breathe as necessary. The music sounds a major ninth lower than written.

\textbf{Myrkr} was written for Richard Haynes who gave the world premiere on 24
October 2015 in Paine Hall on the campus of Harvard University.

\end{textblock}

\end{document}